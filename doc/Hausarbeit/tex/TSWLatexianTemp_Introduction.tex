\documentclass[../Main.tex]{subfiles} 
\begin{document}

\section{Introduction (5p)}
\subsection{Twitter} % State of the Art und so
Twitter ist ein seit 2006 bestehender und heute weltweit genutzter Webdienst zur Versendung von Kurznachrichten (sogenanntes Mikroblogging). Nur ein kleiner Teil der Nachrichten, der sogenannten Tweets, ist in deutscher Sprache verfasst. Die überwiegend genutzten Sprachen sind Englisch, Spanisch, (...). Auch im relativen Vergleich zur Einwohnerzahl zeigt sich, dass in Deutschland, Österreich und der Schweiz wenig getwittert wird, während das Netzwerk etwa in den Niederlanden, Großbritannien, Japan und Indonesien extrem populär ist.

Auch innerhalb Deutschlands ist die räumliche Verteilung versendeter Tweets auf Grundlage mitgesendeter Geodaten ermittelbar. Sie spiegelt größtenteils die Bevölkerungsverteilung wider; Zentren sind vor allem Berlin und das Ruhrgebiet. Allerdings ist -- neben dem Twittern allgemein -- im Besonderen das Mitsenden von Geodaten (sogenanntes Geotagging) im deutschsprachigen Raum eher unpopulär.

Es gibt drei grundsätzlich verschiedene ortsbezogene Informationen, die bei einem Tweet mitgesendet werden können. Keine der Angaben ist für den Nutzer verpflichtend, sodass nicht zu allen Tweets ortsbezogene Daten verfügbar sind.
Zunächst kann der Nutzer in seinem Profil einen Standort eingeben. Er erscheint im JSON-Objekt des Tweets als Feld \textit{location} im Unterobjekt \textit{user.} In unserem Korpus ist zu x\% der Tweets diese Angabe vorhanden. Allerdings werden hier auch sehr gerne Fantasie-Orte eingetragen. Hinzu kommt die mögliche Mehrdeutigkeit und Ungenauigkeit der Angaben, weswegen sie für maschinelle Verwertung praktisch ausscheiden.
Beim Absenden eines Tweets besteht außerdem die Option \textit{Standort hinzufügen}. Dort kann ein eindeutiger, benannter Ort hinzugefügt werden. Im JSON-Objekt erscheint er als Unterobjekt \textit{place}, unter anderem mit Namen, Typ (bspw. "city" für Stadt), Land und einer \textit{bounding box}, einem Rechteck aus vier Geokoordinaten, welches den Ort einschließt.
"Tweets associated with places are not necessarily issued from that location but could also potentially be \textit{about} that location."
Diese Angabe findet sich in unserem Korpus zu x\% der Tweets.
Schließlich ist es möglich, von GPS-fähigen Geräten aus direkt den tatsächlichen Absendeort des Tweets mitzusenden. Die Koordinaten werden als Unterobjekt \textit{geo} im JSON gespeichert. x\% der Tweets in unserem Korpus tragen diese Angaben.


\subsection{General Idea}
Ziel: ungefähre regionale Einordnung eines Tweets innerhalb des deutschssprachigen Raums trotz der o. g. seltenen Geodaten und schlecht benutzbaren Herkunftsangaben %...
Idee: Sprache verrät Herkunft, also sollte aus dem reinen Tweettext die Herkunft ablesbar sein %...

Es existieren hier zwei unterschiedliche Größen. Zum einen gibt es den oder die Orte, an denen ein Twitterer aufgewachsen ist und die ihn sprachlich geprägt haben. Hauptsächlich diese Orte messen wir, wenn wir nach mundartlichen Ausdrücken und Ausdrücken der regionalen Alltagssprache suchen. Auf der anderen Seite steht der momentane, mitunter sehr kurzfristige Aufenthaltsort, von dem aus der Nutzer twittert. Auf Inhaltsebene der Tweets wird er sich eher in ortsbezogenen Begriffen (Ortsnamen, Verkehrsknotenpunkte, Lokalitäten, lokale Ereignisse und Persönlichkeiten etc.) widerspiegeln. Er ist es außerdem, den wir aus den Geodaten von Tweets erfahren. Nun sind Geodaten jedoch die einzigen Daten, die wir zur Evaluierung unserer Ergebnisse verwenden können. Während unser geodatengestützter Ansatz damit recht passend evaluiert werden kann, zielt unser Regiowort-Ansatz speziell auf die Größe \textit{sprachliche Herkunft des Twitterers} ab und wird die Evaluierung daher zwangsläufig mit einem gewissen Handicap absolvieren.

Ansatz:
\begin{itemize}
\item Einteilung des deutschsprachigen Raums in Regionen
\item Machine Learning auf Trainingsdaten aus diesen Regionen
\item Bag-of-words model (Betrachtung von Unigrammen)
\end{itemize}
%...

\subsection{Regional word attempt}
This Regio-approach was inspired by the funny fact that even for such ordinary things like small bakery products or other foods exist a confusing mess or words. Funny means that many of these product got names of towns different from the speakers-town. So a small sausage in Berlin is called "Wiener" whilst in Vienna it is called "Frankfurter". This shows that the use of language esp. the use of lexical items differs from throughout regions. The aim of our regional word approach was to exploit this fact and try to map different words to specific regions and to create a vector-space-models of words for each region. In doing so we should be able to compare new tweets against these models an assign it to one region. 
\subsection{Geo location attempt}
As allready mentioned above german twitter users only seldom allow twitter to store geo data. But nonetheless there is a small number of tweet with these geo data. The Geo-approach uses these data by exactly assign a specific region to each tweet and additionally assign all words in this tweet also to this region. This will be done for all tweets in the train corpus and so finally we created a vector-space-model again - this time based on the exact knowledge of geographical origins of all tweets. 
\subsection{Expectations}
\subsection{Sources, used corpora}
Our work based on the internally so called "Scheffler-Corpus" - a collection containing X-------X tweets from 1.April 2012 to 30.April 2012 collected by using the Twitter-API and which are recognized as german tweets by the langid-module.
\subsection{Regions}
Die Grundüberlegungen, auf denen unsere konkrete Einteilung des deutschen Sprachraumes in Regionen basiert, gab hauptsächlich der Regiowort-Ansatz vor. Die Datenlage im Atlas der deutschen Alltagssprache zeigte uns, in welcher Größenordnung man den Sprachraum auf Basis regionaler Alltagssprache einteilen kann. Auch die konkrete Festlegung der einzelnen Regionen trafen wir beim Sichten der dortigen Daten. Dennoch trug hier auch Twitter als unser Anwendungsbereich ein entscheidendes Kriterium bei: Es sollten Regionen entstehen, in denen jeweils mit einem genügend großen Aufkommen von Tweets zu rechnen war. So zeigte sich in den Daten des Atlas der deutschen Alltagssprache auch eine kleinere Region im Gebiet Saarland/Luxemburg mit charakteristischen Eigenheiten, die wir allerdings wegen zu geringer erwartbarer Menge von Tweets nicht übernahmen.

\end{document}
